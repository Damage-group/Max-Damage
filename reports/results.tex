\section{Results and Conclusion}

The last week we concentrated on our implementation.
Still, we have some interesting results and conclusions. We were also able to find our first pattern.
\newline
We did use our own implementation with a support threshold of 0.3 and a confidence threshold of 0.2 and had the following results.
\newline


introduction\_to\_programming --$>$ introduction\_to\_the\_use\_of\_computers 0.596906

introduction\_to\_programming --$>$ programming\_in\_java 0.598547

introduction\_to\_the\_use\_of\_computers --$>$ introduction\_to\_programming 0.730427

programming\_in\_java --$>$ introduction\_to\_programming 0.816235
\newline

A closer look reveals, that according to our association rules, introduction\_to\_programming implies
programming\_in\_java and the other way around.
Therefore we concluded, that the direction and the order, in which the courses are taken, matters.
We know, that usually the course introduction\_to\_programming is taken before programming\_in\_java.
Therefore we had the feeling, that the confidence measurement is unintuitive.
We started to implement and integrate some other measurement.
Those were:
\begin{itemize}
 \item Interest factor aka Lift
\item IS
\item Mutual Information
\item Certainty factor
\end{itemize}

Those either symmetric or asymmetric measurement will give us a different view on the data in the future.
\newline

We also managed to find our first pattern, which seems to resemble the choices a group of student make.
When we limited our considered courses to the non obligatory ones, we found this interesting group of rules.
\newline


digital\_media\_technology software\_architecture --$>$ the\_metalanguage\_xml 0.814696

digital\_media\_technology software\_design\_java --$>$ the\_metalanguage\_xml 0.846154

software\_processes\_and\_quality --$>$ software\_architecture 0.778862
\newline

One can easily see, that those courses can be easily grouped together. They belong to the same subprogram.
