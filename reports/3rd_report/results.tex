\section{Results and conclustion}
For our results we concentrated on the interesting association rules we found in the last iteration which only consist of non-compulsory courses.
\newline
digital\_media\_technology software\_architecture ⇒  the\_metalanguage\_xml
digital\_media\_technology software\_design\_java ⇒ the\_metalanguage\_xml
software\_processes\_and\_quality ⇒ software\_architecture

The underlying frequent set,  stems from the non-sequential data. Now we aim to find this sets in our sequential data.

Assuming, that all the courses for a given transaction were taken at different semesters, we find the following number of transactions:\\
\newline
digital\_media\_technology software\_architecture the\_metalanguage\_xml:    4
 
digital\_media\_technology software\_design\_java  the\_metalanguage\_xml:    1

software\_processes\_and\_quality software\_architecture:                   22

\newline

Now it can also be, that some of the courses are taken within the same teaching period. So we consider the following cases, course in the same teaching period are indicated with ( ).\\
\newline

(digital\_media\_technology) (software\_architecture the\_metalanguage\_xml):        4

(digital\_media\_technology) (software\_design\_java  the\_metalanguage\_xml):       0 

(software\_processes\_and\_quality software\_architecture):                          19

(digital\_media\_technology software\_architecture) (the\_metalanguage\_xml):        6
 
(digital\_media\_technology software\_design\_java)  (the\_metalanguage\_xml):       5


As we can see Digital Media Technology and Software Architecture courses belong to the same teaching period and judging by students’ choices it does not make any sense to take over different periods. Software Process Quality and Software Architecture courses have higher match count, but this is not really a surprise, as here we deal with only two courses, as opposed to the three. Also the courses can be taken in the same teaching period or spread over different periods.
