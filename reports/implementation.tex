\section{Implementation}
The coding part of the problem 2 started by coding some general structure, because 
it seemed that we are going to code much more during this course. The code is now
separated to following steps: input and data transforming, data processing and algorithms, output handling.
We were able to reuse most of the code implemented during the first week, but the old
code needed some optimizaions. The frequency calculation from transactions ($0 \over 1$ matrix) was re written with matrix operations
reducing the time consumed from about 1 minute to 1 second. Candidate generation was
also fixed resulting $1/4$ decrease in time consumption.\newline

In addition to data handling and transformation code, we implemented apriori algorithm 
for rule generation. The implementation follows almost the pseudo code of the book. 
It was noticed that the algorithm of the book is errorneous. 
The algorithm 6.2 calls the rule generation function described in algorithm 6.3.
The initial parameters of the 6.3 are $f_k \in F_k = \{f \mid f \in \mathrm{frequent\ itemsets\ and\ } |f|=k \}$ 
and $H_k = \{\{a\} \mid a \in f_k \}$. In the algorithm 6.3 the first assigments are:
$k = |f_k|$ and $m = |H_m| = 1$. Then the rest of the function is inside an if clause
testing $k > m + 1$. Assuming that $k=2$ we get $2 > 1+1$ which is clearly false. 
Therefore the pseudo code of the book does not even start when searching for rules in 2-itemsets.
If we assume that $k>2$, then the content of the if clause is executed. The first 
line in the loop assigns $H_{m+1} = \mathbf{apriori-gen}(H_m)$ and $H_{m+1}$ is then processed.
This means that the algorithm does not take into account rules of form $f_k \setminus h_1 \Rightarrow g_1$ where $h_1 \in H_1$.
\newline

Two separate group members that were not implementing the algorithms found a bug in it. 
According to that fact, it seems that not only the coders read the code.
